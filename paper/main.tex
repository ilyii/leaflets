% This must be in the first 5 lines to tell arXiv to use pdfLaTeX, which is strongly recommended.
\pdfoutput=1
% In particular, the hyperref package requires pdfLaTeX in order to break URLs across lines.

\documentclass[11pt]{article}

% Change "review" to "final" to generate the final (sometimes called camera-ready) version.
% Change to "preprint" to generate a non-anonymous version with page numbers.
\usepackage[preprint]{coling}

% Standard package includes
\usepackage{times}
\usepackage{latexsym}

% For proper rendering and hyphenation of words containing Latin characters (including in bib files)
\usepackage[T1]{fontenc}
% For Vietnamese characters
% \usepackage[T5]{fontenc}
% See https://www.latex-project.org/help/documentation/encguide.pdf for other character sets

% This assumes your files are encoded as UTF8
\usepackage[utf8]{inputenc}

% This is not strictly necessary, and may be commented out,
% but it will improve the layout of the manuscript,
% and will typically save some space.
\usepackage{microtype}

% This is also not strictly necessary, and may be commented out.
% However, it will improve the aesthetics of text in
% the typewriter font.
\usepackage{inconsolata}

%Including images in your LaTeX document requires adding
%additional package(s)
\usepackage{graphicx}
\usepackage{amsmath}
\usepackage{multirow}
\usepackage{booktabs}
\usepackage{amssymb}
\usepackage{hyperref}


% If the title and author information does not fit in the area allocated, uncomment the following
%
%\setlength\titlebox{<dim>}
%
% and set <dim> to something 5cm or larger.

\title{Multimodal Information Extraction of Supermarket Leaflets}

\author{Xincheng Liao\textsuperscript{}, Junwen Duan\textsuperscript{}\thanks{\ \ Corresponding author. Email: \href{mailto:jwduan@csu.edu.cn}{jwduan@csu.edu.cn}}, Yixi Huang\textsuperscript{}, Jianxin Wang\textsuperscript{} \\
Hunan Provincial Key Lab on Bioinformatics, School of Computer Science and Engineering, \\
Central South University, Changsha, Hunan, China \\
\texttt{\{ostars, jwduan, yx.huang\}@csu.edu.cn, jxwang@mail.csu.edu.cn} \\
\href{https://github.com/OStars/RUIE}{https://github.com/OStars/RUIE}
}


\begin{document}
\maketitle
\begin{abstract}
TODO
\end{abstract}

\section{Introduction}
\subsection{Motivation}
\subsection{Problem Statement}
\subsection{Structure of Paper}

\section{Related Works}
\indent{\textbf{Deal Detection}.}
\indent{\textbf{Optical Character Recognition}.}
OCR-Model-Driven methods use OCR tools to acquire text
and bounding box information. Subsequently, they rely
on the models to integrate text, layout, and visual data.

\indent{\textbf{Information Extraction}.}

\section{Methodological Overview}

\section{Deal Detection}
    \subsection{Datasets}
    \subsection{Model}
    \subsection{Experiments}

\section{Optical Character Recognition}

\section{Information Extraction}

\section{Application}

\section{Conclusion}
    \subsection{Summary}
    \subsection{Future Work}
    ww

% Bibliography entries for the entire Anthology, followed by custom entries
%\bibliography{anthology,custom}
% Custom bibliography entries only
\bibliography{references}

\end{document}
